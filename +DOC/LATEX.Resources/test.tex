% !TEX TS-program = pdflatex
\documentclass[a4paper]{article}
\usepackage[T1]{fontenc}
\usepackage{lmodern}
\usepackage{mathtools}
\usepackage[sfdefault]{roboto}

\title{LaTeX Beispiel}
\author{ssc}
\date{15.01.2019}

\begin{document}

\maketitle
\tableofcontents

\section{Einführung}
  Hier eine kurze \textit{Einführung}
  Die \textbf{Überschrift} wird automatisch im Inhaltsverzeichnis
  aufgelistet.
  --Willkommen bei \LaTeX{}!

\subsection{Schriftstil wechseln}
  Mit einer "Group" kann der Schriftstil gewechselt werden. \newline
  \begingroup
  \fontseries{t}\selectfont
  Das ist die "Thin" Version der Schrift \newline
  \endgroup
  Das ist wieder die "Light" Version der Schrift.

\subsection{Eine Auflistung}
    \begin{description}
        \item [UI]      Atom One Dark UI
        \item [Syntax]  Atom One Dark Syntax
        \item [Font]    Consolas
    \end{description}

\subsection{Eine Gleichung}
  Gleichung ~\eqref{eq:gauss} ist ein \emph{gaussches Integral}

    \begin{equation}
        \int_{-\infty}^{\infty}\exp(-x^2)\,dx = \sqrt{\pi} \label{eq:gauss}
    \end{equation}

\end{document}
